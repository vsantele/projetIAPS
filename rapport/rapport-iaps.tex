\documentclass[a4paper, 11pt]{article}

\usepackage[utf8]{inputenc}
\usepackage[french]{babel}

\title{INFOB317 : Projet Tour de France}
\author{BERG Thibaut - GAILLARD Matthys - SANTELÉ Victor - SMITH Jonathan}

\begin{document}

\maketitle

\newpage

\section{Introduction}

Dans le cadre du cours INFOB317 : Intelligence artificielle et programmation symbolique, nous avons été amenés à informatiser le jeu du Tour de France en une interface web grâce à React et à y ajouter des fonctionnalités intelligente comme un chatbot auquel on peut poser des questions ou encore une intelligence artificielle jouant le rôle d'adversaire.\newline

Ce projet a été divisé en trois grandes parties :
\begin{itemize}
    \item La conception du jeu et de ses mécaniques
    \item La conception du chatbot
    \item La conception de l'intelligence artificielle adverse\newline
\end{itemize}

Il est évident de préciser que le projet se base principalement sur la conception du jeu et de ses mécaniques. Dessus, nous avons implémenté le chatbot et l'intelligence artificielle adversaire. Ce rapport a pour but de décrire la construction du projet et les techniques mises en place. Celui-ci se déroulera en plusieurs parties : 

\begin{enumerate}
	\item \textbf{Démarche générale} : décrit la démarche générale du groupe sur la réalisation du projet et les décisions qui ont été prisent concernant l'implémentation du jeu.
	\item \textbf{Outils} : présente les outils que nous avons utilisés pour mener à bien le projet et donne les raisons de nos choix.
	\item \textbf{Mécaniques de jeu} : décortique comment nous avons construit la structure du jeu et comment nous avons représenté et implémenté le jeu et ses règles en informatique.
	\item \textbf{Chatbot explicateur} : décrit la construction du chatbot et apporte des informations concernant certaines de ses particularités.
	\item \textbf{Interface} : décrit la construction de l'interface web du jeu et son organisation.
	\item \textbf{Intelligence artificielle adversaire} : décrit la construction de l'intelligence artificielle que le joueur va combattre et son fonctionnement.\newline
\end{enumerate}

Enfin, comme pour tout rapport, nous terminerons par une conclusion dans laquelle nous expliquerons notre ressenti par rapport au projet. Nous expliciterons ce que nous avons trouvé comme points forts et points faibles du projet. Nous aborderons également des pistes pour améliorer le projet.

\newpage

\section{Démarche générale}

\newpage

\section{Outils}

\newpage

\section{Mécaniques de jeu}

\newpage

\section{Chatbot explicateur}

\newpage

\section{Intelligence artificielle adversaire}

\newpage

\section{Interface}

\newpage

\section{Conclusion}

\end{document}